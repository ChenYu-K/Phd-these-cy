% Add your acronyms here

%ABCDEFGHIJKLMNOPQRSTUVWXYZ
%%%%%%%%%%%%%%%%%%%%%%%%%%%%%%%%%%%%%%%
% DO NOT MODIFY FROM HERE ...
\cleardoublepage\phantomsection
\addcontentsline{toc}{chapter}{List of Acronyms and Explanation of Terms}\mtcaddchapter 
\chapter*{List of Acronyms and Explanation of Terms}
\thispagestyle{plain}
\addtocounter{counter}{-1}
% ... TO HERE
%%%%%%%%%%%%%%%%%%%%%%%%%%%%%%%%%%%%%%%

%\chapter*{List of Acronyms and Explanation of Terms}

\section*{Acronyms}

\begin{acronym}

%A
 \acro{AASHTO}{The American Association of State Highway and Transportation Officials}
 \acro{AIJ}{Architectural Institute of Japan}
 \acro{ASD}{Allowable Stress Design}
%B
%C
\acro{CSSS}{Chinese Standard for design of steel structures}
\acro{CPSC}{The U.S. Consumer Product Safety Commission}
\acro{C3D8R}{Three-dimensional eight node solid elements with reduced integration}
%D
\acro{DIC}{Digital Image Correlation}
%E
\acro{ECCS}{European Convention for Constructional Steelwork}
\acro{EDX}{Energy Dispersive X-ray Spectroscopy}
%F
\acro{FEA}{Finite Element Analysis}
\acro{FEM}{Finite Element Methods}
\acro{FBHC}{Friction- and Bearing- Type Hybrid Connections}
%G
%H
\acro{HSB}{High-Strength Bolt}
%I
\acro{iSRR}{Injected Steel-Reinforced Resin}
\acro{IFB}{Interference Fit High-Strength Bolt}
%J
\acro{JSHB}{Japan Specifications for Highway Bridges}
\acro{JRA}{Japan Road Association}
\acro{RTRI}{Japan Railway Technical Research Institute}
\acro{JSRS}{Japanese Design Standards for Railway Structures}
%K
%L
\acro{LRFD}{Load and Resistance Factor Design}

%M
%M
\acro{MBBRB}{Mechanical Bearing Blind Rivet-Bolt}
%O
%P
%Q
%R
\acro{RIBJ}{Resin Injection Bolted Joint}
\acro{RBSM}{Rigid body-spring model}
%S
%T
%U
%V
%W
%X
%Y
%Z

 % ADD MORE ACRONYMS AS NEEDED WITH
 % \acro{example}
 % WHEN YOU USE THE ACRONYM IN THE TEXT
 % USE THE COMMAND \ac{example}
\end{acronym}

\section*{Explanation of Terms}

\begin{acronym}


\acro{Abaqus}{A software suite for finite element analysis}

\acro{Bolt shear yield}{When the von-Mises stress of the bolt-shank cross-section of the shear plan reached the bolt yield strength}

    
\acro{Fasteners}{Generic term for welds, bolts, rivets, or other connecting device}

\acro{Friction type connections}{Also referred to as friction connections, involve joining materials that transmit stress through frictional resistance generated by the inter-material compressive force produced when the jointing materials are tightened using high-strength bolts}

\acro{Hybrid connections}{The connections that combining with different type connections, such as Rivet-HSB, Weld-Rivet, Weld-HSB, IFB-HSB, etc.}

\acro{long bolted joints}{The row number of the joint is over than 8}

\acro{Major slip}{When the joint stiffness changes significantly and the load hardly increases with displacement}

\acro{0.2 mm slip}{The \ac{0.2 mm slip} strength was defined with reference to the Architectural Institute of Japan \cite{AIJ2012AIJStructures} which states that slip occurs when the relative displacement reaches 0.2 mm, where the relative displacement for the main and splice plates is measured to be 10 mm from the end of the main plate. }


\acro{Stiffness}{The slope between the load and the overall displacement of the joint}

\acro{Slip critical}{when the relative displacement reaches 0.2 mm}

\acro{The net cross-sectional yield load}{the equivalent plastic strain appeared at the whole net cross-sectional location of the side at the end of the bolt position}

\acro{The initial stiffness change point}{The time when the entire joint begins to exhibit nonlinear behavior}

\acro{Bearing critical}{when the end bolts of the hybrid bolt joint reached the shear yield strength or the end-bolt holes reached the bearing/shear strength (that is, the end bolt must satisfy Eq.\ref{eq-FV})}

\acro{Bearing critical}{when the relative displacement reaches 0.2 mm}


\end{acronym}