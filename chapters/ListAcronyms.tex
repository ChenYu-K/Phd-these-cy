% Add your acronyms here

%ABCDEFGHIJKLMNOPQRSTUVWXYZ

\renewcommand{\nomname}{List of Symbols and Acronyms}
%%%%%%%%%%%%%%%%%%%%%%%


\nomenclature{$\sigma_{bs}$}{Bearing strength}
\nomenclature{$\sigma_{yp}$}{Yield strength of the main plate}
\nomenclature{$\sigma_{yf}$}{Yield strength of the fastener}

\nomenclature{$\sigma_{ct-m22}$}{Equivalent contact pressure for M22-F10T bolt}
\nomenclature{$\tau_{yf}$}{Shear strength of fastener}
%A
\nomenclature{$A_{cts}$}{specimens size for compact slip test}
\nomenclature{$A_{b}$}{Effective bearing area}
\nomenclature{$A_{s}$}{Effective shear area of fastener shaft}

%B
%C
%D
\nomenclature{$d$}{Nominal diameter of the bolt}
\nomenclature{$d_f$}{Diameter of fastener}
\nomenclature{$d_h$}{Diameter of fastener hole}
\nomenclature{$d_1$}{Diameter of the hole at outermost fastener}
%E
\nomenclature{$e_1$}{The distance between end of plate and the end fastener hole center}
\nomenclature{$e_1$}{edge distance from the centre of a fastener hole to the adjacent edge of any part}
%F
\nomenclature{Fastener}{Generic term for welds, bolts, rivets, or other connecting device.}
\nomenclature{$F_{bs1}$}{Bearing resistance of one fastener}
\nomenclature{$F_{bjr1}$}{Bearing resistance of one fastener under RTRI}
\nomenclature{$F_{b,CS}$}{Design Bearing resistance of CSSS}
\nomenclature{$F_{b,EC3}$}{Design Bearing resistance of Eurocode 3}
\nomenclature{$F_{b,JSHB}$}{Design Bearing resistance of JSHB}
\nomenclature{$F_{b,AASHTO}$}{Design Bearing resistance of AASHTO}
\nomenclature{$F_{vf,CS}$}{Design fastener shaft Shear resistance of CSSS}
\nomenclature{$F_{vf,EC3}$}{Design fastener shaft Shear resistance of Eurocode 3}
\nomenclature{$F_{vf,JSHB}$}{Design fastener shaft Shear resistance of JSHB}
\nomenclature{$F_{vf,AASHTO}$}{Design fastener shaft Shear resistance of AASHTO}

\nomenclature{$F_{vp1,AIJ}$}{Design end plate shear resistance of AIJ with effective cross-sectional area}


\nomenclature{$F_{sh1}$}{Shear resistance of one fastener}
\nomenclature{$F_{slip1}$}{Slip resistance of one fastener}
\nomenclature{$F_{yp}$}{Net cross-section yield resistance of main plate}

\nomenclature{$f_{yf}$}{yield strength stress of fastener}
\nomenclature{$f_{yb}$}{yield strength stress of bolt}
\nomenclature{$f_{yr}$}{yield strength stress of rivet}
\nomenclature{$f_{u}$}{Tensile strength stress for the main plate}
\nomenclature{$f_{uf}$}{Tensile strength stress for fastener}
\nomenclature{$f_{ub}$}{Tensile strength stress for HSB}
\nomenclature{$f_{ur}$}{Tensile strength stress for rivet}

%G
%H
%I
%J
%K
\nomenclature{$k_m$}{material reduction factor to compute bearing resistance per bolt}


%L
\nomenclature{$L_{c}$}{clear distance between fastener holes}

%M
\nomenclature{$m$}{The number of shear plane}
%N
\nomenclature{$N_0$}{initial perload befor loading}
\nomenclature{$N_{cp}$}{Converted preload for compact slip test}
\nomenclature{$N_f$}{Perload of fasteners}
\nomenclature{$N_r$}{Perload of rivets}
\nomenclature{$N_b$}{Perload of boltes}
\nomenclature{$N_{ifb}$}{Perload of interference fit bolts}
%O
%P
%Q
%R
\nomenclature{$r_b$}{Radius of bolt shaft}
\nomenclature{$r_r$}{Radius of rivet shaft}
\nomenclature{$r_h$}{Radius of fastener hole}
%S
%T
\nomenclature{$t$}{Thickness of the main plate}
%U
%V
\nomenclature{$v$}{Poisson's ratio}
%W
\nomenclature{$w$}{Width of the main plate}
%X
%Y
%Z



\printnomenclature

\begin{acronym}

%A
 \acro{AASHTO}{The American Association of State Highway and Transportation Officials}
 \acro{AIJ}{Architectural Institute of Japan}
%B
%C
\acro{CSSS}{Chinese Standard for design of steel structures}
%D
\acro{DIC}{Digital Image Correlation}
%E
\acro{ECCS}{European Convention for Constructional Steelwork}
%F
%G
%H
\acro{HSB}{High-Strength Bolt}
%I

\acro{IFB}{Interference Fit High-Strength Bolt}
%J
\acro{JSHB}{Japan Specifications for Highway Bridges}
\acro{JRA}{Japan Road Association}
\acro{RTRI}{Japan Railway Technical Research Institute}
\acro{JSRS}{Japanese Design Standards for Railway Structures}
%K
%L
%M
%M
\acro{MBBRB}{Mechanical Bearing Blind Rivet-Bolt}
%O
%P
%Q
%R
%S
%T
%U
%V
%W
%X
%Y
%Z

 % ADD MORE ACRONYMS AS NEEDED WITH
 % \acro{example}
 % WHEN YOU USE THE ACRONYM IN THE TEXT
 % USE THE COMMAND \ac{example}
\end{acronym}

