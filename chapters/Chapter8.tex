\chapter{Conclusions and Future research}
\label{ch8}

%%%%%%%%%%%%%%%%%%%%%%%%%%%%%%%%%%%%%%%
% IMPORTANT
\begin{spacing}{1.25} %THESE FOUR
\minitoc % LINES MUST APPEAR IN
\end{spacing} % EVERY
\onehalfspacing % CHAPTER
% COPY THEM IN ANY NEW CHAPTER
%%%%%%%%%%%%%%%%%%%%%%%%%%%%%%%%%%%%%%%

\section{General conclusions}

This study is separated into two parts. The initial part--\RN{1} entails repairing and enhancing aging riveted bridge through the replacement of damaged rivets with high-strength bolts (HSB) using a friction type connection. The second part-- \RN{2} discusses the use of hybrid joints in the construction of new bridges with HSB and interference fit bot (IFB), aim to enhance component strength in response to increased external live loads without enlarging the structure size. Finite element analysis, experimental and numerical analyses are carried out to elucidate the mechanical transfer mechanism of the hybrid joint, the load distribution of bearing and friction, and to propose reasonable fastener configurations methods, the strength calculation formula is also proposed.
%and finally to propose a method to differentiate the limit states of the hybrid joint and to give the strength calculation formula.

Based on the results of our investigation, we have made the following findings and conclusions. \par

\subsection*{Part--\RN{1} : Rivet-HSB hybrid joint}

\textbf{Chapter 3}

In this chapter, the joint part cut out from a 90-year-old riveted bridge's cross-beam. It evaluated the riveted joint surface's aging condition with red lead by microscope observation and chemical analysis. The slip and pressure distribution tests are also conducted to investigate the joint surface's slip coefficient of the riveted joints and the pressure distribution of riveted joints' surface. And the tensile tests were conducted on existing riveted joints to determine the strength and mechanical behavior of the joints when they were partially or fully replaced with high-strength bolts for friction joints. 

Given the results of the present research, the obtained conclusions are as follows: 

\begin{itemize}
    \item Through microscopic observation and FT-IR analysis, the red substance attached to the rivet joint faying surface was identified as red lead. Since there was no presence of iron oxide ((Fe3O4)) components, it can be concluded that the rivet joint surface did not undergo corrosion.

    \item The average slip coefficient determined by the small-scale slip test was 0.274, with a coefficient of variation (CV) of 0.191. The $\mu - 2\sigma$ value is 0.169. The 95\% confidence interval for the slip coefficient is 0.259-0.289. This is significantly lower than the slip coefficient of 0.4 specified by JSCE for high-strength bolt (HSB) friction connections.
    
    \item In the case of the net cross-section yield type, almost the same bearing capacity was obtained when all the rivets were replaced with high-strength bolts and when only the rivets on the outside of the joint were replaced with high-strength bolts. Therefore, if the rivets have not undergone severe corrosion, it is not recommended to replace them for economic and workability reasons.

    
\end{itemize}

\textbf{Chapter 4}

In this chapter, a three-dimensional elasto-plastic finite displacement analysis was conducted using rivet and high-strength bolt arrangements as parameters to elucidate the load transfer mechanism of a joint using a riveted joint and a high-strength bolt friction joint, considering the preload introduced during tightening. In order to focus on the yielding around the rivet holes in the main plate due to the supporting pressure of the rivets, the plate width b was varied while the plate thickness t was kept constant.

The findings of this chapter are as follows.

\begin{itemize}
    
    \item When the outermost rivet is removed, half of the removed rivet load shared will be transferred to the closest rivet,and farther away from the removed rivet, it will be transferred less load to the rivet. When a rivet is replaced with a bolt, closer to the bolt, the rivet's load transfer function is more affected, while the more distant rivets are almost unaffected.

    \item  In hybrid connections, the load transmission three processes: \RN{1} the friction transmission stage, where the load is transmitted almost solely by friction up to the slip resistance $F_{slip}$, \RN{2} the hybrid transmission stage, where the load is transmitted by both bearing and friction up to the bearing load $F_{bya}$, and \RN{3} the plastic deformation stage, where the joint undergoes plastic deformation. Hybrid connections not only increase the initial stiffness of the joint compared to riveted connections but also alleviate the bearing stress on the rivets by transmitting the load through friction to the splice plate.

    \item For rivets-HSBs hybrid connections, this study defines the serviceability limit state as being reached when the plastic region of any rivet hole reaches its diameter range in the FE analysis. The strength proposed by this thesis can be obtained by summing the bearing strength ($n_r d_0 t_m \sigma_{y}$) of the rivets and the slip strength ( $\mu m n_b N_b + \mu m n_r N_r$) of the HSBs.

\end{itemize}


\subsection*{Part--\RN{2} : IFB-HSB hybrid joint}

\textbf{Chapter 5}

In order to solve the issues of reduced slip load and interference with secondary members resulting from long bolted joints, this chapter propose a hybrid joint solution that combines Interference body bolts for bearing-type connection and High-strength hexagon bolt for slip-critical connection to reduce the joint length without sacrificing strength. The finite element analysis was conducted by setting joint length and bolt arrangement as parameters to analyze the mechanical behavior of the hybrid joint.

Based on the results of our investigation, we have made the following findings and conclusions.

\begin{itemize}

\item The R12-O case (a friction type bolted joint) occurs at the major slip, before the tensile force reaches the slip strength ($\mu m n N_0$) for the number of bolts owing to the reduction in bolt preload and uneven load distribution. However, by arranging Interference fit bolts (IFBs) at both ends of the friction type bolted joint (hybrid joint), the joint strength can be effectively augmented since Interference body bolts are unaffected by slip strength and generally have a higher bearing strength than slip strength. Hence, the hybrid joint could maintain the required strength while shortening the joint length.


\item When the 12-row joint is shortened to an 8-row hybrid joint, the 0.2 mm slip load of the R8B2 and R8B4 cases could still meet the slip strength required for 12 columns. In addition, the serviceability limit strengths of R8B2 and R8B4 were 58\% and 23\% higher than those of the 12-row friction joint (R12-O), respectively. In addition, the reduction factor of the slip strength of the long-bolted joint can be ignored because the joint length is reduced. This study demonstrated that it is possible to shorten a long-bolted joint by combining a Interference body bolt at the end of the joint. 

\item It is not recommended to use only one Interference body bolt at each end of a long-bolted joint (over 8-row) in practical engineering design. This is because a bolt may experience local shear yield before net cross-sectional yield occurs, which can decrease the load-bearing capacity of the joint. Therefore, to prevent excessive resistance on a single bolt, at least two Interference body bolts should be used at each end of a long hybrid joint.

\end{itemize}

\textbf{Chapter 6}

This chapter focuses on interference fit bolts (IFBs) assembled at both ends of 10-row friction-type bolted joint to elucidate the slip resistance and deformation of the hybrid joints, and to verify whether the interference fit bolts can be assembled in the friction-type joints to improve the strength of the joint. Tensile tests were performed with three specimens: friction-type bolted joint, hybrid joint, and hybrid joint with non-preload interference fit bolts.

On the basis of the results for the10-row hybrid bolted joint, we arrive at the following findings and conclusions.

\begin{itemize}

\item The overall load-deformation relationship of the hybrid joint remains quasi-linear up to the upper load limit of 1800 kN, although a minor amount of slip occurs. And the total residual deformation of the hybrid joint is only 0.2\% relative to the length $L_j$ of the joint. On the other hand, friction type bolted joint produce a residual deformation of 0.75\% when loaded to 1600kN. Therefore, within the quasi-linear range, it can be expected that the hybrid joint will have a high serviceability limit strength.

\item The observation of the sawed-through cross section showed that the rib portion of the interference fit bolt tail did not provide a good interference fit (the rib was not in contact with the hole wall or the contact area was too small). This resulted in the interference fit bolt being unable to fully resist the portion of load lost due to the slip, causing a minor slip to occur. 

\item The hybrid joint has a small dispersion of the load sharing of each fastener compared to the friction type joint, suggesting that the uneven load distribution and deformation in the joint can be improved when the bearing type connections are properly installed. This allows the inner bolts for friction type connection of the hybrid joint to share more of the load compared to the friction joint for the same load level, reducing the load concentration on the outer fastener.


\end{itemize}



%\subsection*{Limit state of hybrid joint}
\textbf{Chapter 7}

This chapter focuses on long hybrid bolted joints combining bearing and friction type connections. Finite Element Methods (FEM) has been executed to analyzes bolt arrangement, joint length, plate thickness, and material strength. Emphasis is placed on investigating the interactive effects of bearing and friction on joint strength, as well as the reduction in friction and bearing force due to component deformation. This study proposes an effective correction factor for bearing and friction resistance and examines the overall deformation and stress state of the joints to determine the serviceability limit states of hybrid joints.

\begin{itemize}

\item In a hybrid joint, the load is transferred by three processes: \RN{1} a friction transfer phase in which the load is transferred almost exclusively by friction up to a slip capacity, \RN{2} a combined transfer phase in which the load is transferred by bearing  and friction up to serviceability limit state, and \RN{3} a plastic deformation phase in which the joint is plastically deformed afterwards. 

\item The bolt shank shear yield strength can be calculated using Eq. \ref{eq-pvh-2}. For the reduction factor of the friction force, refer to Eq. \ref{eq-as}, where different reduction factors are taken for the cases with two and four fit bolts. As for the reduction factor of the bearing force, since its deviation is small, the lower 99\% CI of 0.76 from the analysis results in this study is taken as the reduction factor (the statistical results are listed in Table \ref{tab-rdfactor}). 

\begin{equation}
    \begin{aligned}
        F_{h,sh,cor-2} = \alpha_s F_s+\alpha_{b} F_b
    \end{aligned}
    \label{eq-pvh-2}
\end{equation}

Where,

\begin{tabularx}{0.95\linewidth}{ l X }
$\alpha_s$ & the reduction factor for the slip resistance;\\ 
$\alpha_b$ & the reduction factor for the bearing resistance of fit bolts, which is taken to be 0.76.\\
\end{tabularx}

\begin{equation}
\alpha_s =
\begin{cases} 
0.89 & \text { if } n_b = 2 \\ 
0.79 & \text { if } n_b = 4
\end{cases}
\label{eq-as}
\end{equation}

\begin{table*}[]
    \centering
    \caption{Reduction factor of the bearing and friction forces}
    \begin{tabular}{@{}cccccccccc@{}}
    \toprule
     & $N_cases$ & Mean & SD & SE & Lower 99\% CI &Upper 99\% CI & Minimum & Median & Maximum \\ \midrule
     $P_b/F_b$ & 26 & 0.774 & 0.027 & 0.006 & 0.76 & 0.79 & 0.728 & 0.775 & 0.83 \\
     Fit bolts: 2 & 14 & 0.79 & 0.021 & 0.006 & 0.775 & 0.803 & 0.759 & 0.786 & 0.83 \\
     Fit bolts: 4 & 9  & 0.75 & 0.018 & 0.006 & 0.735 & 0.765 & 0.728 & 0.745 & 0.785 \\ \midrule
     $P_s/F_s$ & 26 &0.866 & 0.048 &0.01 &0.84 &0.89 &0.775 &0.89 &0.919 \\
     Fit bolts: 2 & 14 & 0.9 & 0.011 & 0.003 & 0.892 & 0.908 & 0.878 & 0.901 & 0.919 \\
     Fit bolts: 4 & 9  & 0.812 & 0.026 & 0.009 & 0.79 & 0.834 & 0.775 & 0.807 & 0.848 \\
     \bottomrule 
    \end{tabular}
    \label{tab-rdfactor}
\end{table*}



% \item Due to the shear yielding of the fasteners, the fastener shaft produces a gross cross-section plastic region and therefore the preload force is significantly reduced $\alpha_v$. In addition, for the front end of the joint, due to the influence of the additional bending moment, the connection plate produces a surface profile change and therefore gaps appear in the contact surfaces, which results in a loss of friction. In such cases, the friction resistance of the relevant fasteners should be reduced (in the case of \#1 fastener at the front end, the friction resistance should not be taken into account).

% \item For the calculation of bearing yielding or shear yielding, it is necessary to consider the reduction factor $\beta_{ls}$ due to the unevenness of the load sharing. As the joint proceeds to the second stage, it can be assumed that the friction of all fasteners is resisted, so the reduction factor for uneven load sharing is not need to considered. In addition, this reduction factor $\beta_{ls}$ actually replaces the reduction factor for long joints, and if this reduction factor is introduced into the calculation, it is not necessary to consider the reduction factor for long joints in the SLS resistance calculation.



\end{itemize}





\section{Future research}

\begin{itemize}

%\renewcommand\labelitemi{\faSearch}

\item Fatigue strength of hybrid joints

The fatigue strengths of friction and bearing connections with high-strength bolts are different, so the fatigue strength of hybrid connections also needs to be discussed.

\item Long-term repeated loading of hybrid joints

Since bearing links are based on the elastic-plastic deformation of the steel to transfer the load, the creep of the hybrid link needs to be confirmed by repeated loading over a long period of time after setting the bearing limit state.

\item Effect of different faying surfaces condition

Hybrid joints with different joint surfaces also need to be discussed because of the different effects of the degree of slip undercutting brought about by the different joint surfaces.

\item Effect of and the dynamic friction

So far the discussion on friction joints has been limited to the slip coefficient, i.e. the static friction coefficient, when calculating hybrid joints, the actual joints are in a state of dynamic friction, therefore the effect of dynamic friction on hybrid joints needs to be discussed in more depth, especially as the mechanism of static-dynamic friction changes for different connection surfaces.

\item Reduction factor for ultimate limit state.

For the limit states, it is considered sufficient to follow the existing methods of strength calculation and classification. The only difference is that since all fasteners do not enter the bearing connection at the same time, fasteners that enter the bearing connection at the beginning may fail prematurely due to early entry into the bearing state, and therefore the strength of the limit states of the hybrid connection may be considered to require a reduction factor.

\end{itemize}
