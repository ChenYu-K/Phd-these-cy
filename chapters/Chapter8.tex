\chapter{Conclusions and Future research}
\label{ch8}

%%%%%%%%%%%%%%%%%%%%%%%%%%%%%%%%%%%%%%%
% IMPORTANT
\begin{spacing}{1.25} %THESE FOUR
\minitoc % LINES MUST APPEAR IN
\end{spacing} % EVERY
\onehalfspacing % CHAPTER
% COPY THEM IN ANY NEW CHAPTER
%%%%%%%%%%%%%%%%%%%%%%%%%%%%%%%%%%%%%%%

\section{General conclusions}

This study is separated into two parts. The initial part--\RN{1} entails repairing and enhancing aging riveted bridge through the replacement of damaged rivets with high-strength bolts (HSB) using a friction type connection. The second part-- \RN{2} discusses the use of hybrid joints in the construction of new bridges with HSB and interference fit bot (IFB), aim to enhance component strength in response to increased external live loads without enlarging the structure size. Finite element analysis, experimental and numerical analyses are carried out to elucidate the mechanical transfer mechanism of the hybrid joint, the load distribution of bearing and friction, and to propose reasonable fastener configurations methods, the strength calculation formula is also proposed.
%and finally to propose a method to differentiate the limit states of the hybrid joint and to give the strength calculation formula.

Based on the results of our investigation, we have made the following findings and conclusions. \par

\subsection*{Part--\RN{1} : Rivet-HSB hybrid joint}

\begin{itemize}

    \item When the outermost rivet is removed, half of the removed rivet load shared will be transferred to the closest rivet,and farther away from the removed rivet, it will be transferred less load to the rivet. When a rivet is replaced with a bolt, closer to the bolt, the rivet's load transfer function is more affected, while the more distant rivets are almost unaffected.

    \item In the case of the bearing yield first type, where the critical service condition is determined by the bearing, all rivet holes yielded when the load-total displacement relationship of the entire joint began to exhibit nonlinear behavior, and the load at that time was consistent with the equation which is calculated as the sum of the sliding capacity and bearing pressure capacity of each fastener. In the case of the pure sectional yield-precedence type where the service limit state is determined by the net sectional yield, the net sectional yield is superior to the deformation due to bearing pressure, and the net sectional yield load can be evaluated using the net sectional yield capacity equation.

    \item  In the net-section yield first type case, the net sectional yield capacity of the case with fastener position 1 as a high-strength bolt increased approximately 10\% and the initial stiffness also increased compared to the fully riveted case because the load was transferred to the bracing plate by frictional force. In the case with high-strength bolts at fastener positions 2 and 3, the joint had higher initial stiffness than the fully riveted case, but the difference in net cross-sectional yield strength was smaller.

    \item In the case of the bearing yield first type, the difference in the support load between the all-rivet case and the combined-joining case is small regardless of the bolt arrangement, but the combined-joining case is affected by frictional forces.

\end{itemize}


\subsection*{Part--\RN{2} : IFB-HSB hybrid joint}

\textbf{Chapter 5}

\begin{itemize}

\item When the 12-row joint is shortened to an 8-row hybrid joint, the 0.2 mm slip load of the R8B2 and R8B4 cases could still meet the slip strength required for 12 columns. In addition, the serviceability limit strengths of R8B2 and R8B4 were 58\% and 23\% higher than those of the 12-row friction joint (R12-O), respectively. In addition, the reduction factor of the slip strength of the long-bolted joint can be ignored because the joint length is reduced. This study demonstrated that it is possible to shorten a long-bolted joint by combining a Interference body bolt at the end of the joint. 

\item It is not recommended to use only one Interference body bolt at each end of a long-bolted joint (over 8-row) in practical engineering design. This is because a bolt may experience local shear yield before net cross-sectional yield occurs, which can decrease the load-bearing capacity of the joint. Therefore, to prevent excessive resistance on a single bolt, at least two Interference body bolts should be used at each end of a long hybrid joint.

\item The slip load of a 10-row hybrid joint assembled with four interference fit bolts at both ends is approximately 7\% higher than that of a friction joint, and if the initial preload of the bolts is taken into account as inconsistent, the hybrid joint is approximately 10\% higher than the friction joint when the slip load is divided by the respective calculated slip resistance.

\end{itemize}

\textbf{Chapter 6}

\begin{itemize}

\item The overall load-deformation relationship of the hybrid joint remains quasi-linear up to the upper load limit of 1800 kN, although a minor amount of slip occurs. And the total residual deformation of the hybrid joint is only 0.2\% relative to the length $L_j$ of the joint. On the other hand, friction type bolted joint produce a residual deformation of 0.75\% when loaded to 1600kN. Therefore, within the quasi-linear range, it can be expected that the hybrid joint will have a high serviceability limit strength.

\item The observation of the sawed-through cross section showed that the rib portion of the interference fit bolt tail did not provide a good interference fit (the rib was not in contact with the hole wall or the contact area was too small). This resulted in the interference fit bolt being unable to fully resist the portion of load lost due to the slip, causing a minor slip to occur. 

\item Long friction type bolted joint occurs at the major slip before the tensile force reaches the designed slip strength owing to the reduction in bolt preload and uneven load distribution. However, by arranging Interference body bolts at both ends of the friction type bolted joint (hybrid joint), the joint strength can be effectively augmented since Interference body bolts are unaffected by slip strength and generally have a higher bearing strength than slip strength. Hence, the hybrid joint could maintain the required strength while shortening the joint length.

\item The hybrid joint has a small dispersion of the load sharing of each fastener compared to the friction type joint, suggesting that the uneven load distribution and deformation in the joint can be improved when the bearing type connections are properly installed. This allows the inner bolts of the hybrid joint to share more of the load compared to the friction joint for the same load level, reducing the load concentration on the outer fastener.

\end{itemize}



\subsection*{Limit state of hybrid joint}

\begin{itemize}

\item The service limit states can be categorized as: a.Shear yield limit state for fastener shaft $F_{hv}$, b. Bearing yield limit state for main plate $F_{hb}$, c. Net cross-section yield limit state $F_y$. The calculation equation shall taken as: 

\begin{table}[htbp]
\centering
\caption*{Calculate equation for hybrid joint on the SLS}
\begin{tabular}{@{}lcc@{}}
\toprule
\multicolumn{3}{c}{Serviceability limit state}                                                  \\ \midrule
\multicolumn{1}{c}{} & \begin{tabular}[c]{@{}c@{}}If b1 arranged \\ fastener for bearing\end{tabular} & \begin{tabular}[c]{@{}c@{}}If \\ not\end{tabular} \\ \cmidrule(l){2-3} 
\begin{tabular}[c]{@{}l@{}}Shear yield resistance\\ for fastener shaft $F_{hv}$\end{tabular} &$(n_f+ \alpha_v(n_b-1)) F_s + \beta_{ls} n_b F_{vy1}$ & $(n_f+ \alpha_v n_b) F_s + \beta_{ls} n_b F_{vy1}$ \\
\begin{tabular}[c]{@{}l@{}}Bearing yield resistance\\ for main plate $F_{hb}$ \end{tabular}   & $(n-1) F_s + \beta_{ls} n_b F_{by1}$ & $ n F_s + \beta_{ls} n_b F_{by1}$ \\
\begin{tabular}[c]{@{}l@{}}Net cross-section yield\\ resistance $F_y$\end{tabular}   & \multicolumn{2}{c}{$ (w-d_0) t f_y$} \\ \bottomrule
\end{tabular}
\end{table}

\item In a hybrid joint, the load is transferred by three processes: \RN{1} a friction transfer phase in which the load is transferred almost exclusively by friction up to a slip capacity, \RN{2} a combined transfer phase in which the load is transferred by bearing  and friction up to serviceability limit state, and \RN{3} a plastic deformation phase in which the joint is plastically deformed afterwards. 

\item Due to the shear yielding of the fasteners, the fastener shaft produces a gross cross-section plastic region and therefore the preload force is significantly reduced $\alpha_v$. In addition, for the front end of the joint, due to the influence of the additional bending moment, the connection plate produces a surface profile change and therefore gaps appear in the contact surfaces, which results in a loss of friction. In such cases, the friction resistance of the relevant fasteners should be reduced (in the case of \#1 fastener at the front end, the friction resistance should not be taken into account).

\item For the calculation of bearing yielding or shear yielding, it is necessary to consider the reduction factor $\beta_{ls}$ due to the unevenness of the load sharing. As the joint proceeds to the second stage, it can be assumed that the friction of all fasteners is resisted, so the reduction factor for uneven load sharing is not need to considered. In addition, this reduction factor $\beta_{ls}$ actually replaces the reduction factor for long joints, and if this reduction factor is introduced into the calculation, it is not necessary to consider the reduction factor for long joints in the SLS resistance calculation.

\end{itemize}





\section{Future research}

\begin{itemize}

\renewcommand\labelitemi{\faSearch}

\item Fatigue strength of hybrid joints

The fatigue strengths of friction and bearing connections with high-strength bolts are different, so the fatigue strength of hybrid connections also needs to be discussed.

\item Long-term repeated loading of hybrid joints

Since bearing links are based on the elastic-plastic deformation of the steel to transfer the load, the creep of the hybrid link needs to be confirmed by repeated loading over a long period of time after setting the bearing limit state.

\item Effect of different faying surfaces condition

Hybrid joints with different joint surfaces also need to be discussed because of the different effects of the degree of slip undercutting brought about by the different joint surfaces.

\item Effect of and the dynamic friction

So far the discussion on friction joints has been limited to the slip coefficient, i.e. the static friction coefficient, when calculating hybrid joints, the actual joints are in a state of dynamic friction, therefore the effect of dynamic friction on hybrid joints needs to be discussed in more depth, especially as the mechanism of static-dynamic friction changes for different connection surfaces.

\item Reduction factor for ultimate limit state.

For the limit states, it is considered sufficient to follow the existing methods of strength calculation and classification. The only difference is that since all fasteners do not enter the bearing connection at the same time, fasteners that enter the bearing connection at the beginning may fail prematurely due to early entry into the bearing state, and therefore the strength of the limit states of the hybrid connection may be considered to require a reduction factor.

\end{itemize}
