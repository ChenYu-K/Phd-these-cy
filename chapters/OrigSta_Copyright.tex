%%%%%%%%%%%%%%%%%%%%%%%%%%%%%%%%%%%%%%%
% DO NOT MODIFY FROM HERE ...
\cleardoublepage\phantomsection
\addcontentsline{toc}{chapter}{Preface}\mtcaddchapter 
\chapter*{Preface}
\addtocounter{counter}{-1}
% ... TO HERE
%%%%%%%%%%%%%%%%%%%%%%%%%%%%%%%%%%%%%%%

% EDIT FROM HERE
% Example text

This research started in April 2021 with the goal to obtain the Ph.D degree in Urban Engineering (Bridge Engineering Lab.) at the Osaka City University. This research investigated the load transfer mechanism, resistance and limit states of hybrid joint with friction type and bearing type connections. The results are based on both experimental and FE analyses.

%论文分别参考了欧洲的eurocode3,美国的aashto,中国和日本的设计标准,对承压式连接进行了深度的讨论。承压式连接能够提供更高的强度,然而极限设计法中各国对于承压式连接的设计都非常暧昧,除了对于强度极限状态的设计有明确的规定,对于使用极限状态大多数标准都没有给出明确的定义。日本的设计标准虽然对承压式连接给出了设计方法,然而这是基于ASD方法的延用,对于接头的变形等没有做出充分的讨论。因此本研究首先着重于讨论了承压式连接,然后讨论了其与摩擦式连接共同作用时接头的变形以及强度。
The thesis provides an in-depth discussion of bearing connections with reference to Eurocode 3 in Europe, AASHTO in the USA, and design standards in China and Japan, respectively. The bearing connection can provide higher bearing resistance, however, the limit state design method is very ambiguous for the design of bearing type connection in all countries, except for the design of ultimate (strength) limit state, most of the standards do not give a clear definition for the serviceability limit state. Although the Japanese design standard gives a design method for bearing type connection in SLS, it is based on the extension of the ASD method, and the deformation of the joint is not sufficiently discussed. Therefore, this study firstly focuses on the bearing connection, and then discusses the deformation and bearing resistance of the joint when it acts together with the friction connection.

%本论文的部分内容是作者在代尔夫特理工大学进行研究访问期间撰写的,在进行关于注射螺栓的共同研究时,获得了许多关于混合连接的灵感。
Parts of this thesis were written during the author's research visit to the Delft University of Technology, where many inspirations for hybrid connections were gained while carrying out joint research on injection bolts.

This thesis is completed under overleaf based on \LaTeX, the template of the article is referenced and modified from [\hyperlink{https://www.overleaf.com/latex/templates/msc-report-template-eee-imperial-college-london-v1-dot-0-3/qtkpngktpwpw}{MSc Report Template:} EEE Imperial College London v1.0.3], following follow the CC BY 4.0 share-alike agreement, the modified template as well as its content are open source shared in my GitHub project (\faGithub : \hyperlink{https://github.com/ChenYu-K/Phd-these-cy}{ChenYu-K}) \footnote{\faGithub : \url{https://github.com/ChenYu-K/Phd-these-cy}}.

Papers related to this thesis can be viewed via Orcid \orcidlink{0000-0002-1187-4761} or \hyperlink{https://www.researchgate.net/profile/Yu-Chen-505}{Researchgate (\texttt{Yu-Chen-505})} \footnote{\aiResearchGate: {\href{https://www.researchgate.net/profile/Yu-Chen-505}{\texttt{https://www.researchgate.net/profile/Yu-Chen-505}}}}, and any questions about the data or the research methodology can be directed to the authors \footnote{\faEnvelope:{\href{mailto:chen.yu@skiff.com}{\texttt{ chen.yu@skiff.com}}}}.

%我在代尔夫特研究访问期间完成了绝大部分内容,

%此文章在overleaf基于latex下完成,文章的模板参考并修改自[MSc Report Template EEE Imperial College London v1.0.3],遵循遵循CC BY 4.0共享协议,修改后的模板以及其内容均开源共享在我的github项目中。

%这篇文章的英文绝大部分由AI(Chatgpt-3.5Turbo)翻译并进行了文法矫正.
%在此声明没有用AI进行文本内容生成,所有的文本内容均由著者由中文编写翻译或者直接通过英语撰写出来。

% 联系方式以及相关论文

%support by MM bridge, JST spring project.

%%%%%%%%%%%%%%%%%%%%%%%%%%%%%%%%%%%%%%%
% DO NOT MODIFY FROM HERE ...
\cleardoublepage\phantomsection
\addcontentsline{toc}{chapter}{Declaration}\mtcaddchapter 
\chapter*{Declaration}
\addtocounter{counter}{-1}
% ... TO HERE
%%%%%%%%%%%%%%%%%%%%%%%%%%%%%%%%%%%%%%%

% EDIT FROM HERE
% Example text

\subsection*{Copyright}

The copyright of this thesis (Ph.D Thesis: "Study on Limit States of Hybrid Joint With Friction and Bearing Type Connection" © Mar. 2024 by Yu Chen) rests with the author and is made available under a Creative Commons Attribution license \hyperlink{https://creativecommons.org/licenses/by/4.0/}{CC BY 4.0} \ccby. 

Researchers are free to copy and redistribute the material in any medium or format for any purpose, even commercially. Adapt — remix, transform, and build upon the material for any purpose, even commercially. The licensor cannot revoke these freedoms as long as you follow the license terms.

\subsection*{Generative AI}

It is hereby declared that no Generative AI was used to generate the text content, and that all text content was written and translated by the authors from Chinese \& Japanese or written directly in English. However, the \texttt{ChatGPT-3.5 turbo} as well as \texttt{DeepL} were used on this thesis to translate and correct the grammar.