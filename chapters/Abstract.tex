\onehalfspacing 
% Steel bridges constructed between 1860 and 1955 were predominantly connected with riveted joints. In Japan, the period of cast iron started in 1897 which later shifted towards steel industry. After 1955, advancements in steel technology enabled the production of high-strength steel. As a result of the bolt's ease of construction and its reliable mechanical transmission mechanism, it quickly supplanted the rivet as the prevailing joining method.

Rivets have been a reliable way to transfer mechanical force due to their ability to resist sliding through physical shear, resulting in increased strength and energy absorption. However, its installation poses difficulties due to the need for high precision drilling, leading to its gradual replacement by high-strength bolts. Recently, repairing deteriorated rivet bridges has become a concern, with replacing rivets with high-strength bolts being the most common solution in Japan. Other options, such as interference fit bolts, have also been developed as alternatives. Combining these fasteners with friction type connections can provide high strength while overcoming the drawbacks of friction type connections, such as large slip, Slip coefficient dependence and low energy absorption, without significantly affecting constructability since only some of the bearing fasteners need to be installed.

This study is separated into two part. The first part entails repairing and enhancing aging riveted bridge through the replacement of damaged rivets with high-strength bolts using a friction connection. The second part discusses the use of hybrid joints in the construction of new bridges. Aim to enhance component strength in response to increased external live loads without enlarging the structure size. Finite element analysis, experimental and numerical analyses are carried out to elucidate the mechanical transfer mechanism of the hybrid joint, the load distribution of bearing and friction, and to propose reasonable fastener configurations methods, and finally to propose a method to differentiate the limit states of the hybrid joint and to give the strength calculation formula.

In this study, it was found that regardless of the installation position of the fasteners a bearing connection with lower rigidity will share less force compared to a friction connection with higher rigidity, which implies that the difference in the installation position of the fasteners results in a different mechanical transfer mechanism. The FE analysis and experiment results showed that 12-row hybrid joints could improve the slip load by approximately 20\% compared with 12-row friction-bolted joints. A hybrid joint with Interference body bolts can effectively improve its strength and shorten the length of the slip-critical bolted joint.

In a hybrid joint, the load is transferred by three processes: (1) a friction transfer phase in which the load is transferred almost exclusively by friction up to a slip capacity, (2) a combined transfer phase in which the load is transferred by bearing pressure and friction up to a bearing pressure load, and (3) a plastic deformation phase in which the joint is plastically deformed afterwards. Finally, this study defined the serviceability limit state of the hybrid joint as bolt shear yield of the bolt shank, the bearing yield or the net cross-sectional yield of the plate. Additionally, we presented an equation to evaluate the bolt shear yield and bearing strength of a hybrid joint, a load sharing factor for two type connections are also presented.

%载荷分担系数

\textbf{Keywords:} Friction type bolted connection, Bearing type bolted connection, riveted joint, Hybrid joint.