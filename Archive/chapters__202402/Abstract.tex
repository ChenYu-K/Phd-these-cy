

\onehalfspacing 
% Steel bridges constructed between 1860 and 1955 were predominantly connected with riveted joints. In Japan, the period of cast iron started in 1897 which later shifted towards steel industry. After 1955, advancements in steel technology enabled the production of high-strength steel. As a result of the bolt's ease of construction and its reliable mechanical transmission mechanism, it quickly supplanted the rivet as the prevailing joining method.

Rivets have been a reliable way to transfer tensile load due to their no large sliding behaviour like friction type connections through physical contact bearing transfer mechanism, resulting in increased strength and energy absorption. However, its installation poses difficulties due to the need for high precision drilling, leading to its gradual replacement by high-strength bolts. Recently, repairing deteriorated rivet bridges has become a concern, with replacing rivets with high-strength bolts being the most common solution in Japan. Other options, such as interference fit bolts, have also been developed as alternatives. Combining these fasteners with friction type connections can provide high strength while overcoming the drawbacks of friction type connections, such as large slip, Slip coefficient dependence and low energy absorption, without significantly affecting constructability since only some of the bearing fasteners need to be installed.
%铆钉一直是传递机械力的可靠方式,因为它们能够通过物理剪切抵抗滑动,从而提高强度和能量吸收。 但由于需要高精度钻孔,其安装存在困难,逐渐被高强度螺栓所取代。 最近,修复老化的铆钉桥已成为一个问题,用高强度螺栓替换铆钉是日本最常见的解决方案。 其他选项,例如过盈配合螺栓,也已被开发作为替代方案。 将这些紧固件与摩擦式连接相结合,可以提供高强度,同时克服摩擦式连接的缺点,例如滑移大、滑移系数依赖性和能量吸收低,而不会显着影响可施工性,因为只需要安装部分轴承紧固件。

This study is separated into two part. The first part entails repairing and enhancing aging riveted bridge through the replacement of damaged rivets with high-strength bolts using a friction connection. The second part discusses the use of hybrid joints in the construction of new bridges with HSB and interference fit bot (IFB), aim to enhance component strength in response to increased external live loads without enlarging the structure size. Finite element analysis, experimental and numerical analyses are carried out to elucidate the mechanical transfer mechanism of the hybrid joint, the load distribution of bearing and friction, and to propose reasonable fastener configurations methods, and proposed a strength calculation formula.

%and finally to propose a method to differentiate the limit states of the hybrid joint and to give the strength calculation formula.
%本研究分为两部分。 第一部分需要通过使用摩擦连接用高强度螺栓替换损坏的铆钉来修复和增强老化的铆接桥。 第二部分讨论混合接头在新桥梁建设中的使用。 旨在增强部件强度,以应对增加的外部活荷载,而不扩大结构尺寸。 通过有限元分析、实验和数值分析,阐明了混合接头的机械传递机制、轴承和摩擦力的载荷分布,并提出了合理的紧固件配置方法,最终提出了区分混合接头极限状态的方法。 并给出混合接头的强度计算公式。

In this study, it was found that regardless of the installation position of the fasteners a bearing connection with lower rigidity will share less force compared to a friction connection with higher rigidity, which implies that the difference in the installation position of the fasteners results in a different mechanical transfer mechanism. The FE analysis and experiment results showed that 12-row hybrid joints could improve the slip load by approximately 20\% compared with 12-row friction-bolted joints. A hybrid joint with Interference body bolts can effectively improve its strength and shorten the length of the slip-critical bolted joint.
%在这项研究中,发现无论紧固件的安装位置如何,与具有较高刚度的摩擦连接相比,具有较低刚度的轴承连接将分担较小的力,这意味着紧固件安装位置的差异会导致 不同的机械传输机构。 有限元分析和实验结果表明,与12排摩擦螺栓接头相比,12排混合接头可将滑动载荷提高约20%。 采用干涉体螺栓的混合接头可以有效提高其强度并缩短滑动临界螺栓接头的长度。

In a hybrid joint, the load is transferred by three processes: (1) a friction transfer phase in which the load is transferred almost exclusively by friction up to a slip capacity, (2) a combined transfer phase in which the load is transferred by bearing pressure and friction up to a bearing pressure load, and (3) a plastic deformation phase in which the joint is plastically deformed afterwards. 

Finally, this study defined the serviceability limit state of the hybrid joint as bolt shear yield of the bolt shank, the bearing yield or the net cross-sectional yield of the plate. Additionally, we presented an equation to evaluate the bolt shear yield and bearing strength of a hybrid joint, a load sharing factor for two type connections are also presented.
%在混合万向节中,载荷通过三个过程传递:(1) 摩擦传递阶段,其中载荷几乎完全通过摩擦力传递至滑动能力,(2) 组合传递阶段,其中载荷通过 承受压力和摩擦直至达到承受压力载荷,以及(3)塑性变形阶段,其中接头随后发生塑性变形。 最后,本研究将混合接头的正常使用极限状态定义为螺栓杆的螺栓剪切屈服、承载屈服或板的净截面屈服。 此外,我们提出了一个评估混合接头的螺栓剪切屈服强度和承载强度的方程,还提出了两种类型连接的载荷分配系数。

%载荷分担系数

\textbf{Keywords:} Friction type connection, Bearing type connection, riveted joint, Hybrid joint, Limit states.